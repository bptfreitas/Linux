\documentclass[10pt,a4paper,oneside]{exam}

\usepackage[provaresumida]{cefetex}

\professor{Bruno Policarpo Toledo Freitas}
\curso{T�cnico em Inform�tica Integrado ao Ensino M�dio}
\disciplina{Sistemas Operacionais}
\nomedaprova{Prova Bimestral 2}

\begin{document}
	
\maketitle

\renewcommand{\lstlistingname}{Solu��o}

\begin{questions}
	
\question[2] Diga o que os seguintes comandos UNIX significam:

	\begin{parts}
		\begin{minipage}{\textwidth}
			\part chmod 777 arquivo.txt
			\begin{solutionorlines}[1cm]
			\end{solutionorlines}
		\end{minipage}
	
		\begin{minipage}{\textwidth}
			\part chown joao:bruno arquivo.txt
			\begin{solutionorlines}[1cm]
			\end{solutionorlines}
		\end{minipage}
	
		\begin{minipage}{\textwidth}
			\part kill -9 1234
			\begin{solutionorlines}[1cm]
			\end{solutionorlines}
		\end{minipage}
	
		\begin{minipage}{\textwidth}
			\part grep "bruno" arquivo.txt
			\begin{solutionorlines}[1cm]
			\end{solutionorlines}
		\end{minipage}
	
		\begin{minipage}{\textwidth}
			\part sudo su
			\begin{solutionorlines}[1cm]
			\end{solutionorlines}
		\end{minipage}	
		
		\begin{minipage}{\textwidth}
			\part ps -u bruno 
			\begin{solutionorlines}[1cm]
			\end{solutionorlines}
		\end{minipage}
	
		\begin{minipage}{\textwidth}
			\part cat arquivo1.txt arquivo2.txt $|$ sort $>$ arquivo3.txt
			\begin{solutionorlines}[1cm]
			\end{solutionorlines}
		\end{minipage}	
	
	\end{parts}

\question[2] Para cada express�o regular abaixo, d� duas strings de exemplo que elas capturam

	\begin{parts}
		
		\begin{minipage}{\textwidth}
			\part $[abc]*abc$
			\begin{solutionorlines}[1cm]
			\end{solutionorlines}
		\end{minipage}
	
		\begin{minipage}{\textwidth}
			\part $[d-z]+abc$
			\begin{solutionorlines}[1cm]
			\end{solutionorlines}
		\end{minipage}
	
		\begin{minipage}{\textwidth}
			\part $[a-z]*abc\{2,3\}$
			\begin{solutionorlines}[1cm]
			\end{solutionorlines}
		\end{minipage}	
	
		\begin{minipage}{\textwidth}
			\part $ \hat{} \  [A-Z0-9]+[a-z0-9]*\$$
			\begin{solutionorlines}[1cm]
			\end{solutionorlines}				
		\end{minipage}
	


%		\part $\^[A-Za-z]+\.[A-Za-z0-9]$
	\end{parts}


\baselineskip=2\baselineskip
\begin{minipage}{\textwidth}	
\question[2] Quais s�o os 4 tipos de redirecionamento? O que eles fazem? Cite um exemplo de utiliza��o para cada um deles.

	\begin{solutionorlines}[3cm]
	\end{solutionorlines}
\end{minipage}



\begin{minipage}{\textwidth}
	\question[2] Explique, sucintamente, as seguintes pastas do sistema de arquivos UNIX:\\

	\begin{parts}

		\part /bin  \hrulefill
		
		\part /home \hrulefill

		\part /dev \hrulefill

		\part /usr \hrulefill

		\part /lib \hrulefill			

		\part /tmp \hrulefill
				
		\part /var \hrulefill
				
	\end{parts}
\end{minipage}
	

\begin{minipage}{\textwidth}		
	\question[2] Explique, sucintamente, o que o seguinte shell script faz:

	\lstinputlisting[language=sh]{codigo1.sh}

	\begin{solutionorlines}[3cm]
	\end{solutionorlines}
\end{minipage}

\end{questions}


\end{document}
